% preamble
\documentclass[11pt,a4paper]{article}

\usepackage[utf8]{inputenc}
\usepackage{amsmath}
\usepackage{amsfonts}
\usepackage{amssymb}
\usepackage[margin=1in, footskip=0.25in]{geometry} % para configurar los margenes
\usepackage[shortlabels]{enumitem} % para usar enumerate con letras
\usepackage{commath} % para usar \norm{} 
\usepackage{mathabx} % para usar \check{} para versores
\usepackage{siunitx} % para usar \ang{}
\usepackage{stmaryrd} % para usar \sslash

\title{\'Algebra y Geometr\'ia Anal\'itica} % crea el titulo

\date{}

\begin{document} % inicia el documento

\maketitle % imprime el titulo

\begin{enumerate}
\setcounter{enumi}{0} % inicia el contador

% Ejercicio 1
\item Recuerde, ¿qu\'e es un vector? Seleccione y marque la opci\'on correcta. Un vector es:

\begin{enumerate}[a)]
\item Dos puntos en el plano \textit{xy}
\item Un segmento de recta entre dos puntos.
\item Un segmento de recta dirigido que tiene m\'odulo, direcci\'on y sentido.
\end{enumerate}

% Ejercicio 2
\item Graficar los siguientes puntos en $\mathbb{R}^{2}$ o en $\mathbb{R}^{3}$ seg\'un corresponda. Para ello dibuje los ejes correspondientes (\textit{x, y}) o (\textit{x, y, z}). \par

$A = (3, 4)$, $B = (0, -2)$, $C = (-1, -5)$, $D = (2, 3, 4)$, $E = (-2, -3, 5)$, $F = (-2, 3, -4)$ 

% Ejercicio 3
\item A partir del concepto de vector de posici\'on (o localizaci\'on), grafique los vectores de posici\'on correspondientes a los puntos del apartado anterior.

% Ejercicio 4
\item Haciendo uso de GeoGebra, grafique los puntos mencionados en el apartado 2) y sus correspondientes vectores de posici\'on mencionados en el apartado 3).

% Ejercicio 5 
\item Dados los siguientes vectores $\vec{u}_{1} = (-1, 8)$, $\vec{u}_{2} = (2, 7)$ y $\vec{v} = (2, -5)$, repres\'entelos gr\'aficamente. Luego obtenga los siguientes vectores:

\begin{enumerate}[a)]
\item $\vec{w} = \vec{u}_{1} + \vec{u}_{2}$
\item $\vec{u}_{1} + \vec{v}$ 
\item $\vec{u}_{2} + \vec{v} + \vec{w}$
\item $-\vec{u}_{1}$
\item $\vec{v}$ - $\vec{u}_{2}$
\item $\vec{u}_{2}$ - $\vec{v}$
\end{enumerate}

% Ejercicio 6
\item Dados los siguientes vectores $\vec{u} = (6, -4, 6)$, $\vec{v} = (1, 5, 4)$ y $\vec{w} = (3, 2, 5)$, repres\'entelos gr\'aficamente. Luego obtenga los siguientes vectores:

\begin{enumerate}[a)]
\item $\vec{u} + \vec{v}$ 
\item $\vec{u} + \vec{v} + \vec{w}$
\item $-\vec{u}$
\item $\vec{v}$ - $\vec{u}$
\item $\vec{u}$ - $\vec{v}$
\end{enumerate}

% Ejercicio 7
\item Dados los siguientes vectores $\vec{u} = (-1, 2, 3)$ y $\vec{v} = (0, -2, 5)$ y el escalar $k = -2$, encontrar los siguientes m\'odulos: \par

\begin{enumerate}[a)]
\item $\norm{\vec{u}}$
\item $\norm{\vec{u} + \vec{v}}$ y comp\'arelo con $\norm{\vec{u}} + \norm{\vec{v}}$
\item $\norm{k * \vec{u}}$ y comp\'arelo con $\abs{k} * \norm{\vec{u}}$ 
\end{enumerate}

% Ejercicio 8
\item ¿Cu\'al es la caracter\'istica principal de un vector unitario? ¿Qu\'e otro nombre recibe? ¿C\'omo convertimos un vector en un vector unitario? Encuentra el versor o vector unitario de los siguientes vectores: 

\begin{enumerate}[a)]
\item $\vec{u} = (-2, 5, 4)$
\item $\vec{v} = (0, -6, 0)$
\item $\vec{w} = (2, 2, 2)$
\end{enumerate}

% Ejercicio 9
\item ¿C\'omo se escribir\'ian los versores $\check{i}$, $\check{j}$ y $\check{k}$ en funci\'on de sus componentes? Dados los siguientes vectores $\vec{u} = 2 \check{i} - \check{j} + 3 \check{k}$ y $\vec{v} = 3 \check{j} + 2 \check{k}$, realice las siguientes operaciones y grafique los resultados:

\begin{enumerate}[a)]
\item $2\vec{u} + \vec{v}$
\item $-\vec{u} + 2\vec{v}$ 
\end{enumerate}

% Ejercicio 10
\item Dados los puntos $P = (1, -1, 5)$ y $Q = (-3, 1, 3)$. Encuentre los vectores de posici\'on de los puntos $P$ y $Q$ y el vector $\vec{PQ}$.

% Ejercicio 11
\item ¿Qu\'e son los \'angulos directores de un vector? ¿Cu\'al es la relaci\'on fundamental? Encuentre los \'angulos directores de los siguientes vectores $\vec{u} = (1, 3, 4)$, $\vec{v} = (1, 1, 1)$, $\vec{w} = (2, 3, 0)$, $\vec{t} = (0, 5, 2)$. Verifique la relaci\'on fundamental para los \'angulos directores encontrados anteriormente.

% Ejercicio 12
\item Dados los siguientes vectores, $\vec{u} = (-1, 3, 4)$, $\vec{v} = -2 \check{j} + 5 \check{k}$ y $\vec{w} = (3, 0, 4)$, realice los siguientes productos escalares:

\begin{enumerate}[a)]
\item $\vec{u} * \vec{v}$
\item $\vec{v} * \vec{u}$
\item $(\vec{u} + \vec{v}) * \vec{w}$
\item $(\vec{u} * \vec{w}) + (\vec{v} * \vec{w})$
\end{enumerate}

% Ejercicio 13
\item Calcule el \'angulo entre los vectores $\vec{u}$ y $\vec{v}$, $\vec{u}$ y $\vec{w}$ del punto anterior.

% Ejercicio 14
\item A partir de los vectores $\vec{u}$ y $\vec{v}$ del inciso 12) y recordando las condiciones de paralelismo y perpendicularidad entre vectores, verifique cu\'al de las siguientes condiciones cumplir\'ian:

\begin{enumerate}[a)]
\item $\vec{u}$ y $\vec{v}$ son ortogonales.
\item $\vec{u}$ y $\vec{v}$ son paralelos.
\item $\vec{u}$ y $\vec{v}$ no son paralelos ni ortogonales.
\end{enumerate}

% Ejercicio 15
\item Encuentre un vector ortogonal al vector $\vec{t} = (-1, 4, 3)$. ¿Cu\'antos vectores ortogonales podr\'ian encontrarse para el vector $\vec{t}$?

% Ejercicio 16
\item Dados los vectores $\vec{u} = (3, 2, 4)$ y $\vec{v} = (2, -1, 5)$ encuentre el vector proyecci\'on de $\vec{u}$ en la direcci\'on de $\vec{v}$. Calcule el m\'odulo de la proyecci\'on.

% Ejercicio 17
\item Dados los siguientes pares de vectores, realice los productos vectoriales correspondientes:

\begin{enumerate}[a)]
\item $\vec{u} = (1, -1, 7)$ y $\vec{v} = (2, -3, 0)$
\item $\vec{u} = (2, -1, 3)$ y $\vec{v} = (4, -2, 6)$
\item Entre los versores $\check{i}$ y $\check{j}$
\item Entre los versores $-\check{j}$ y $\check{k}$
\end{enumerate}

% Ejercicio 18
\item Calcule el siguiente producto mixto:

\begin{enumerate}[a)]
\item $\check{i} * (\check{j} \times \check{k})$
\item $(1, 1, 1) * ((-2, 3, 5) \times (-1, 4, 2))$
\end{enumerate}

% Ejercicio 19
\item Investigue en la bibliograf\'ia recomendada sobre el producto mixto entre vectores, y a continuaci\'on conteste las siguientes preguntas:

\begin{enumerate}[a)]
\item ¿Qu\'e operaciones y en qu\'e orden deben efectuarse para calcular un producto mixto entre vectores?
\item ¿Cu\'al es la expresi\'on de c\'alculo del producto mixto entre vectores?
\item Escriba y ejemplifique las propiedades del producto mixto entre vectores.
\item ¿C\'omo se establece si tres vectores del espacio tridimensional son coplanares?
\item ¿Qu\'e interpretaci\'on geom\'etrica admite el valor absoluto del producto mixto entre vectores? Definici\'on, propiedades e interpretaci\'on geom\'etrica.
\end{enumerate}
 
% Ejercicio 20 
\item Debate con tus compa\~neros la veracidad o falsedad de las siguientes afirmaciones. Para que tu respuesta sea v\'alida debes justificarla adecuadamente, por lo cual puedes emplear definiciones, ejemplos, gr\'aficos, c\'alculos.

\begin{enumerate}[1)]
\setcounter{enumi}{0} % inicia el contador

\item El vector $\vec{u} = (1, 1, 1)$ es un versor.
\item Los vectores $\vec{u} = (1, 0, 1)$ y $\vec{v} = (0, 1, 0)$ son paralelos.
\item Los vectores $\vec{u} = (1, 2,-1)$ y $\vec{v} = (-2, -4, 2)$ son perpendiculares.
\item Los \'angulos que forma el versor fundamental $\check{i}$ en $\mathbb{R}^{3}$ con los ejes coordenados (\textit{x, y, z}) valen $\ang{0}$, $\ang{90}$ y $\ang{180}$ respectivamente.
\item Un versor determina una y solo una direcci\'on en el espacio.
\item Las componentes de un versor son los \'angulos directores de su direcci\'on.
\item Una direcci\'on se establece un\'ivocamente mediante un versor o su opuesto.
\item Todo versor es vector y todo vector es versor.
\item La distancia entre dos puntos del plano o del espacio es igual al m\'odulo del vector que los tiene
por origen y extremo respectivamente.
\item M\'odulo, longitud, magnitud son vocablos sin\'onimos.
\item Un vector en el plano se define por sus dos componentes, o mediante su m\'odulo y su argumento.
\item Para establecer una direcci\'on en el plano se requiere conocer sus dos \'angulos directores.
\item El vector nulo de $\mathbb{R}^{2}$ es perpendicular a cualquier otro vector del plano.
\item Un punto del plano o del espacio puede ubicarse mediante sus coordenadas cartesianas o
mediante su vector posici\'on.
\item Un vector perpendicular al eje \textit{z} es paralelo a cualquier vector del plano \textit{xy}.
\item Si un vector es paralelo al eje \textit{y}, tambi\'en es paralelo al eje \textit{z}.
\item Num\'ericamente, las coordenadas cartesianas de un punto (de $\mathbb{R}^{2}$ o $\mathbb{R}^{3}$) son iguales a las
componentes del vector posici\'on del punto.
\item Cuando normalizamos un vector estamos hallando un versor paralelo al vector dado.
\item $\norm{\vec{u} \times \vec{v}} = \norm{\vec{v} \times \vec{u}}$ 
\item El producto escalar no es conmutativo.
\item Si $\vec{u} \times \vec{v} = \vec{0}$, entonces $\vec{u} \sslash \vec{v}$
\item La distancia entre cualquier punto (de $\mathbb{R}^{2}$ o $\mathbb{R}^{3}$) al origen es igual al m\'odulo de su vector posici\'on.
\item Si $\vec{u} = k \vec{v}$, (donde $\vec{u}$ y $\vec{v}$ son vectores de $\mathbb{R}^{2}$ o $\mathbb{R}^{3}$ y $k$ un n\'umero real) entonces $\norm{\vec{u}} > \norm{\vec{v}}$
\item Si un vector es m\'ultiplo de otro entonces ambos son perpendiculares.
\item Si $\vec{w}$ es combinaci\'on lineal de $\vec{u}$ y de $\vec{v}$, entonces ambos son perpendiculares.
\item El m\'odulo del producto vectorial $\vec{u} \times \vec{v}$ es igual al \'area del paralelogramo que tiene por lados a $\vec{u}$ y $\vec{v}$
\item Si $\vec{u} = (u_{1}, 1, 3)$ y $\norm{u}^{2} = 11$ entonces $u_{1} = 1$ o $u_{1} = -1$
\item Todos los vectores del plano o del espacio, que tienen la misma direcci\'on, el mismo sentido y el mismo m\'odulo, pero que difieren en su punto origen, pertenecen a una misma familia. Se dice que son equivalentes.
\item El vector nulo (de $\mathbb{R}^{2}$ o $\mathbb{R}^{3}$) no tiene direcci\'on ni sentido y su m\'odulo vale cero.
\item El vector nulo (de $\mathbb{R}^{2}$ o $\mathbb{R}^{3}$) se representa geom\'etricamente por el punto origen del sistema de
coordenadas cartesianas.
\end{enumerate}

\end{enumerate} % cierra la enumeracion de los ejercicios

\end{document} % cierra el documento